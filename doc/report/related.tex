\section{Related Work}
\label{sec:related-work}

%% Describe related work here.
% the following related works are more from end-host perspective.

We first review to possible tools and methodologies that will be valuable for our projects in this section.

There are sizable tools existing for Internet measurement, and they differ in the characteristics they measure. To name a few, {\em ping} is famous to test reachability and measure round-trip time (RTT). {\em traceroute} \cite{jacobson1989traceroute} and {\em tcptraceroute} \cite{toren2001tcptraceroute} could tell the routing path as well as RTT. King \cite{gummadi2002king} measures the end-to-end latency using recursive DNS queries. Other than time characteristics, {\em pathchar} \cite{jacobson1997pathchar} measures the hop-by-hop bandwidth and \cite{jain2002pathload} tells the available bandwidth of a path. \cite{paxson2002experiences} provides dedicated hardware for measurements, only affiliated  users can access. Therefore in \cite{spring2003scriptroute}, Spring {\em et al.}\, seeks to build a generic platform for ordinary users to conduct measurements from remote vantage points.

On top of these tools, many large Internet monitor projects (such as pingER \cite{matthews2000pinger}) are built. And researchers are also trying to infer the Internet performance (such as \cite{downey1999using} utilized {\em pathchar} tool to estimate Internet link characteristics) and security \cite{paxson1999bro} based on the measurement results. 

With the advent of cloud computing and mobile computing, it becomes important to characterize their performance. \cite{kandula2009nature, benson2010network} analyzes traffic patterns in data center, and \cite{huang2011mobiperf} is an ongoing project in measuring mobile end-host network performance. But it's unclear how the Internet's performance is when in comparison.

Paxson's work in \cite{paxson1997measurements, paxson1997end, paxson2004strategies} serves as guidelines in our design and implementation of all measurement experiements.

%%% Local Variables: 
%%% mode: latex
%%% TeX-master: "main"
%%% End: 
