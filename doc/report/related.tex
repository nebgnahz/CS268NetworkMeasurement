\section{Related Work}
\label{sec:related-work}

%% Describe related work here.
% the following related works are more from end-host perspective.

\subsection{Tools and Infrastructures}
\label{sec:tools}

There are a number of existing tools for Internet measurement; they primarily differ in the network characteristics they measure. {\em ping} is the popular tool used to test reachability and measure round trip time (RTT). {\em traceroute} \cite{jacobson1989traceroute} and {\em tcptraceroute} \cite{toren2001tcptraceroute} reveal the routing path as well as the RTT. King \cite{gummadi2002king} measures the end-to-end latency of arbitrary end-hosts using recursive DNS queries. Turbo King (T-King \cite{leonard2008turbo}) proposes an alternative methodology for generating more accurate measurement results than King. {\em pathchar} \cite{jacobson1997pathchar} measures the hop-by-hop bandwidth and \cite{jain2002pathload} reports the available bandwidth of a path. \cite{paxson2002experiences} provides dedicated hardware for measurements that only affiliated users can access. In \cite{spring2003scriptroute}, Spring {\em et al.}\, seeks to build a generic platform for ordinary users to conduct measurements from remote vantage points.

On top of these tools, many large Internet monitoring projects (such as pingER \cite{matthews2000pinger} and iPlane \cite{madhyastha2006iplane}) have been built. A number of infrastructures that serve as experimental testbeds are also available to conduct large scale measurements. Among these, GENI \cite{elliott2008geni} and PlanetLab \cite{chun2003planetlab} are two active projects.

% Furthermore, researchers are also trying to infer Internet performance and security \cite{paxson1999bro} based on measurement results (\cite{downey1999using} utilized {\em pathchar} tool to estimate Internet link characteristics). 

In our measurement study, we do not seek to introduce new measurement strategies or infrastructure. Instead, we derive our strategy from the King, T-King, {\it traceroute}, {\it ping}, and research-oriented infrastructure Planet Lab.

\subsection{Measurements}
\label{sec:measurements}

With the advent of cloud computing and mobile computing, it has become increasingly important to characterize the performance of the data center and mobile networks. 

Both \cite{benson2010network, kandula2009nature} analyze traffic patterns in data centers. These measurement results help in developing an understanding of the nature of traffic in an operational data center (especially for those with out direct access) and guides many system designs. VL2 \cite{greenberg2009vl2} is one such example; the validation of Valiant Load Balancing on flow level is based on the small flow traffic pattern in a data center. Our measurement is not meant to address the latency within data center. Instead, we start from the experience of end users, and partition the latency of queries to cloud-based services into the WAN and DC portions. Such a partition can provide additional insight as to the importance of deadlines in the data center as has been previously studied in \cite{wilson2011better}. By comparing the portion of time spent in the WAN and the DC, we aim to understand which is the primary element that determines an end-host's experienced RTT.

There are also several works on mobile networks \cite{xu2011cellular, huang2011mobiperf} that try to infer the networking conditions by distributing their applications to a large numbers of users. These work provides insight on a larger scale, but seldom touch on the latency dynamics for a single user over a longer time period. Our measurement aims to fill this gap. In addition, most of our cellular network measurements occur within the 4G-LTE network, a relatively recent mobile networking development.


%\subsection{Measurement Methodologies}
%\label{sec:meas-meth}

% Early Vern Paxson's work in \cite{paxson1997measurements, paxson1997end, paxson2004strategies} serves as guidelines in our design and implementation of all measurement experiements.

%%% Local Variables: 
%%% mode: latex
%%% TeX-master: "main"
%%% End: 
