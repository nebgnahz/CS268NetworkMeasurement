
\documentclass{sig-alternate}

\usepackage{times}
\usepackage{graphics}

\usepackage{subfigure}
\usepackage{booktabs}
\usepackage{colortbl}
\usepackage{tabularx}
\usepackage{color}
\usepackage{xspace}
\usepackage{hyperref}    % Creates hyperlinks from ref/cite
\hypersetup{pdfstartview=FitH}
\usepackage{graphicx}    % For importing graphics
\usepackage{url}         %

\hypersetup{%
pdftitle={An Internet Latency Measurement Study}, pdfauthor={Ben Zhang, Ahir Reddy}, pdfkeywords={Internet, Latency, Measurement}, bookmarksnumbered, pdfstartview={FitH}, colorlinks,
citecolor=black, filecolor=black, linkcolor=black, urlcolor=black,
breaklinks=true,}

\renewcommand{\arraystretch}{1.2} % Space out rows in tables

\newcommand{\ml}[1]{{\color{green} {\it ML: #1}}}

% No space between bibliography items:
\let\oldthebibliography=\thebibliography
  \let\endoldthebibliography=\endthebibliography
  \renewenvironment{thebibliography}[1]{%
    \begin{oldthebibliography}{#1}%
      \setlength{\parskip}{0ex}%
      \setlength{\itemsep}{0ex}%
  }%
  {%
    \end{oldthebibliography}%
  }

%\pagenumbering{arabic}  % Arabic page numbers for submission.  Remove this line to eliminate page numbers for the camera ready copy

\begin{document}

% use this command to override the default ACM copyright statement
% (e.g. for preprints). Remove for camera ready copy.
%\toappear{Submitted for review to IPSN 2012.}
% \conferenceinfo{ConfName} {Date, Location}
% \CopyrightYear{Year} 
% \crdata{978-1-4503-1227-1/12/04} 
% \clubpenalty=10000 
% \widowpenalty = 10000


% to make various LaTeX processors do the right thing with page size
\special{papersize=8.5in,11in}
\setlength{\paperheight}{11in}
\setlength{\paperwidth}{8.5in}
\setlength{\pdfpageheight}{\paperheight}
\setlength{\pdfpagewidth}{\paperwidth}

\title{Measurements and Analysis of Internet}

\author{
{Ben Zhang, Ahir Reddy}\\
\affaddr{University of California, Berkeley}\\
%\affaddr{}\\
\email{benzh@eecs.berkeley.edu, ahirreddy@berkeley.edu}
}

\maketitle

\begin{abstract}
% Abstract for measurement project.
Understanding the state-of-art Internet performance and locating the bottle-neck is critical to both academic research and industrial systems. Though various metrics (connectivity, available Bandwidth, etc.) are to be evaluated, in this project, we focus on the latency of current Internet (as time of year 2013) -- including Wide Area Network (WAN), Data Center, and Celullar Networks. 

We employed King/T-King method proposed in the literature to measure the latency between two arbitrary hosts in the Internet. The correlation of latency with geographical information identifies the gap between existing network performance and speed-of-light limit. Such data serves as the starting point of analyzing wide area network performance. With increasing demand of putting service into the cloud, the latency experienced by end-user is not only depending on the WAN, but also the data center performance. We choose Google Search as a case study to understand the composition of time in typical query conversations. The third study we did is about celullar networks, specifically about the latest 4G-LTE network in Bay Area deployed by AT\&T. We used {\it traceroute} to test the latency of this new flat IP architecture in LTE networks. And the data suggests fairly good traffic engineering work within AT\&T's celular backbone.

\end{abstract}

%\category{C.2.1}{Computer-Communication Networks}{Network Architecture and Design}[Wireless communication]
%\keywords{Collection, CTP, Sensor Network, Routing}

% \category{B.0}{Hardware}{General}
% \category{B.4}{Hardware}{Input/Output \& Data Communication} 
% \category{H.4.m}{Information Systems Applications}{Miscellaneous}

% \terms{Design, Experimentation, Measurement} 
% \keywords{Localization, Magneto-Inductive, Tracking, Virtual Zone}

\section{Introduction}
\label{sec:introduction}

The measurement of Internet has become increasingly important and challenging due to its complexity, large-scale and variability. A numbers of tools have been proposed to accomplish this task (see Sec.\,\ref{sec:related-work} for a complete review). In this project, our primary goal is not to introduce any novel network measurement tool; instead, we use existing measurement techniques to characterize the performance of current Internet. We base our work on previous investigations, and try to elucidate the changes in measurement strategies and outcomes. Specifically, the cloud computing service model and mobile computing have re-shaped the Internet in unanticipated ways. The impact of these new models on end-to-end latency and measurement techniques have never been formally studied. We propose the questions listed bellow as motivations for our study.

\begin{itemize}
\item How does network latency vary among different kinds of end hosts -- generic end users(including mobile/wireless end hosts), regular web servers, and cloud service from data centers?
\item How current Internet latency performs in comparison to the speed-of-light limit?
\item How wide-area networks and data centers contribute to end-to-end latency experienced by end users?
\item How the latency is affected by edge caching of content, especially on mobile networks?
\item What are the characteristics of time-series analysis for latency on a daily/weekly/yearly basis?
\end{itemize}

Our initial approach will consist of re-implementing and re-producing previous measurement works in the context of current Internet, following the guidelines outlined in \cite{paxson2004strategies}. One major strategy we intend to re-implement is King \cite{gummadi2002king}, a strategy which leverages the DNS infrastructure to measure end-to-end latency of arbitrary end hosts. To guarantee deliverables, we have a outlined an initial timetable for our study below(see Table.\,\ref{tab:plan}).

\begin{table}
  \centering
  \begin{tabular}{ c|p{4cm} }
    \hline
    Period & Task \\
    \hline
    Late Feburary & Literature review \\
    March & Re-implementation and initial measurements \\
    April & Revision/correction of any possible methodology flaws \\
    Early May & Wrap up and report \\ 
    \hline
  \end{tabular}
  \label{tab:plan}
  \caption{timeline for project}
\end{table}

%%% Local Variables: 
%%% mode: latex
%%% TeX-master: "main"
%%% End: 

\section{Related Work}
\label{sec:related-work}

Describe related work here.


\section{Expected Results}
\label{sec:results}

We expect to have results in the form of actual measurement tools and data, as well as analysis tools and reproducible analysis output. The dataset will consist of latency measurements between end hosts along with meta-data (the DNS query, Geo-IP resolution, time of testing, environment of testing, etc.). Our analysis will compare latency to time and physical location, and separate WAN latency from datacenter latency.

%%% Local Variables: 
%%% mode: latex
%%% TeX-master: "main"
%%% End: 


% \section{Acknowledgments}

\bibliographystyle{abbrv}
\bibliography{Measurement}

\end{document}
