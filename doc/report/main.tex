\documentclass{sig-alternate}

\usepackage{times}
\usepackage{graphics}

\usepackage{subfigure}
\usepackage{booktabs}
\usepackage{colortbl}
\usepackage{tabularx}
\usepackage{color}
\usepackage{xspace}
\usepackage{hyperref}    % Creates hyperlinks from ref/cite
\hypersetup{pdfstartview=FitH}
\usepackage{graphicx}    % For importing graphics
\usepackage{url}         %

\hypersetup{%
pdftitle={An Internet Latency Measurement Study}, pdfauthor={Ben Zhang, Ahir Reddy}, pdfkeywords={Internet, Latency, Measurement}, bookmarksnumbered, pdfstartview={FitH}, colorlinks,
citecolor=black, filecolor=black, linkcolor=black, urlcolor=black,
breaklinks=true,}

\renewcommand{\arraystretch}{1.2} % Space out rows in tables

\newcommand{\ml}[1]{{\color{green} {\it ML: #1}}}

% No space between bibliography items:
\let\oldthebibliography=\thebibliography
  \let\endoldthebibliography=\endthebibliography
  \renewenvironment{thebibliography}[1]{%
    \begin{oldthebibliography}{#1}%
      \setlength{\parskip}{0ex}%
      \setlength{\itemsep}{0ex}%
  }%
  {%
    \end{oldthebibliography}%
  }

%\pagenumbering{arabic}  % Arabic page numbers for submission.  Remove this line to eliminate page numbers for the camera ready copy

\begin{document}

% use this command to override the default ACM copyright statement
% (e.g. for preprints). Remove for camera ready copy.
%\toappear{Submitted for review to IPSN 2012.}
% \conferenceinfo{ConfName} {Date, Location}
% \CopyrightYear{Year} 
% \crdata{978-1-4503-1227-1/12/04} 
% \clubpenalty=10000 
% \widowpenalty = 10000


% to make various LaTeX processors do the right thing with page size
\special{papersize=8.5in,11in}
\setlength{\paperheight}{11in}
\setlength{\paperwidth}{8.5in}
\setlength{\pdfpageheight}{\paperheight}
\setlength{\pdfpagewidth}{\paperwidth}

\title{Measurements and Analysis of Internet}

\author{
{Ben Zhang, Ahir Reddy}\\
\affaddr{University of California, Berkeley}\\
%\affaddr{}\\
\email{benzh@eecs.berkeley.edu, ahirreddy@berkeley.edu}
}

\maketitle

\begin{abstract}
% Abstract for measurement project.
Understanding the state-of-art Internet performance and locating the bottle-neck is critical to both academic research and industrial systems. Though various metrics (connectivity, available Bandwidth, etc.) are to be evaluated, in this project, we focus on the latency of current Internet (as time of year 2013) -- including Wide Area Network (WAN), Data Center, and Celullar Networks. 

We employed King/T-King method proposed in the literature to measure the latency between two arbitrary hosts in the Internet. The correlation of latency with geographical information identifies the gap between existing network performance and speed-of-light limit. Such data serves as the starting point of analyzing wide area network performance. With increasing demand of putting service into the cloud, the latency experienced by end-user is not only depending on the WAN, but also the data center performance. We choose Google Search as a case study to understand the composition of time in typical query conversations. The third study we did is about celullar networks, specifically about the latest 4G-LTE network in Bay Area deployed by AT\&T. We used {\it traceroute} to test the latency of this new flat IP architecture in LTE networks. And the data suggests fairly good traffic engineering work within AT\&T's celular backbone.

\end{abstract}

%\category{C.2.1}{Computer-Communication Networks}{Network Architecture and Design}[Wireless communication]
%\keywords{Collection, CTP, Sensor Network, Routing}

% \category{B.0}{Hardware}{General}
% \category{B.4}{Hardware}{Input/Output \& Data Communication} 
% \category{H.4.m}{Information Systems Applications}{Miscellaneous}

% \terms{Design, Experimentation, Measurement} 
\keywords{Internet Latency, Measurement, Turbo King, PlanetLab, traceroute}

\section{Introduction}
\label{sec:introduction}

% why do measurement
The measurement of Internet has become increasingly challenging due to its complexity, large-scale, heterogeneity and variability. However, the importance is also increasing because an understanding of the performance of the Internet architecture will provide the insight necessary to design a better architecture, which is either incrementally deployable or a clean slate design. Depending on figure of interest, there exist many metrics to characterize the performance -- connectivity, available bandwidth, latency, etc. Various tools have been proposed to conduct these measurements (see Sec.\,\ref{sec:related-work} for a complete review). This work does not aim to introduce a revolutionary methodology to conduct measurements. Instead, we base our work on many existing tools and available infrastructure. We will discuss both our methodology and an analysis of the measurements we have collected.

% split up to three parts
Our work study is composed of three distinct parts: the latency in the wide-area network,  latency as experienced by end-users when interacting with the data center, and latency within cellular networks.

The goal in measuring WAN latency is to develop and understanding of how latency is related to distance. The traditional model used to calculate the latency between end-hosts is to sum propagation delay, transmission delay, queuing delay and processing delay \cite{kurose2001computer}. The propagation delay is determined by the length of cables (for wired networks), and will have the lower bound, which we refer to ``speed-of-light limit'' in this paper. The transmission delay and processing delay depend on the speed of routers (or switches) and the size of the packet. The general trend is that these delays are getting smaller. Nonetheless, queuing delay has a lot of variation depending on the traffic load; in the worst case, packets may even be dropped. In our study we aim to uncover the current state of latency in the WAN given the new and recent developments in networking hardware and software. Furthermore, we seek to make a comparison to the theoretical speed-of-light limit. In our study we have adopted the King/Turbo-King \cite{gummadi2002king, leonard2008turbo} methods and deployed measurement nodes to PlanetLab \cite{chun2003planetlab} servers located globally. In a continuous study spanning more than 10 days we have obtained approximately 5 million measurements for analysis.

In the second section of our study, we analyze cloud services and data centers. More specifically, we have chosen to perform a study of Google Search due to its popularity and Google's promise of delivering timely responses to end-users. Furthermore, we are motivated to use Google Search because of the fact that the return page from Google includes an estimate of time spent inside the data center. By inspecting the TCP trace of a query conversation, we found out that this estimated time is a fairly accurate portrayal of time spent within the data center. Because each transaction involves several rounds of data transmission, the round-trip time (RTT) will influence the overall query time. We extract Google's IP address from the packet trace using \texttt{tcpdump} and utilize \texttt{ping} to query this IP to measure the network latency. By examining this data from various vantage points (deployed to PlanetLab servers), we would partition the latency and provide insights for network optimizations.

The final aspect of our study is an analysis of cellular networks, and in particular the relatively new and growing 4G-LTE deployments. By adopting a flat IP architecture, carriers promise to reduce the network latency and decouple radio access from core network evolution. To measure the real performance of the network, we use \texttt{traceroute} to examine the latency on each hop and the routing stability within the cellular backbone.

% summarize the contribution
To summarize, we make the following contributions in this measurement study:
\begin{itemize}
\setlength{\itemsep}{1pt}
\setlength{\parskip}{0pt}
\setlength{\parsep}{0pt}
\item We present measurements of the wide-area network, the fraction of round-trip time spent within the data center, and latencies within the cellular network backbone.
\item We provide the methodology and analysis of our collected measurements.
\end{itemize}

% rest of the paper
In the rest of this paper, we will first describe the related works in Sec.\,\ref{sec:related-work}. And then we detail our measurements and analysis for WAN, DC and cellular networks in Sec.\,\ref{sec:latency-wan}, Sec.\,\ref{sec:latency-DC} and Sec.\,\ref{sec:latency-in-cellular} respectively. We discuss our experience in conducting this measurements in Sec.\,\ref{sec:discussion}. Sec.\,\ref{sec:conclusion} concludes the paper and discusses some of the future work.

% In the rest of this paper, we will first describe the related works in Sec.\,\ref{sec:related-work}. And then detail our measure methodology in Sec.\,\ref{sec:methodology}. The experiment setup is in Sec.\,\ref{sec:experiment}, which is followed by the analysis of the data in Sec.\,\ref{sec:analysis}. We then briefly discuss a few questions regarding our measurements in Sec.\,\ref{sec:discussion}. Sec.\,\ref{sec:conclusion} concludes the paper and discusses some of the future work.

%%% Local Variables: 
%%% mode: latex
%%% TeX-master: "main"
%%% End: 

\section{Related Work}
\label{sec:related-work}

%% Describe related work here.
% the following related works are more from end-host perspective.

\subsection{Tools and Infrastructures}
\label{sec:tools}

There are a number of existing tools for Internet measurement; they primarily differ in the network characteristics they measure. \texttt{ping} is the popular tool used to test reachability and measure round trip time (RTT). \texttt{traceroute} \cite{jacobson1989traceroute} and \texttt{tcptraceroute} \cite{toren2001tcptraceroute} reveal the routing path as well as the RTT. King \cite{gummadi2002king} measures the end-to-end latency of arbitrary end-hosts using recursive DNS queries. Turbo King (T-King \cite{leonard2008turbo}) proposes an alternative methodology for generating more accurate measurement results than King. \texttt{pathchar} \cite{jacobson1997pathchar} measures the hop-by-hop bandwidth and \cite{jain2002pathload} reports the available bandwidth of a path. \cite{paxson2002experiences} provides dedicated hardware for measurements that only affiliated users can access. In \cite{spring2003scriptroute}, Spring {\it et al.}, seeks to build a generic platform for ordinary users to conduct measurements from remote vantage points.

On top of these tools, many large Internet monitoring projects (such as pingER \cite{matthews2000pinger} and iPlane \cite{madhyastha2006iplane}) have been built. A number of infrastructures that serve as experimental testbeds are also available to conduct large scale measurements. Among these, GENI \cite{elliott2008geni} and PlanetLab \cite{chun2003planetlab} are two active projects.

% Furthermore, researchers are also trying to infer Internet performance and security \cite{paxson1999bro} based on measurement results (\cite{downey1999using} utilized \texttt{pathchar} tool to estimate Internet link characteristics). 

In our measurement study, we do not seek to introduce new measurement strategies or infrastructure. Instead, we derive our strategy from the King, T-King, \texttt{traceroute}, \texttt{ping}, and research-oriented infrastructure PlanetLab.

\subsection{Measurements}
\label{sec:measurements}

With the advent of cloud computing and mobile computing, it has become increasingly important to characterize the performance of the data center and mobile networks. 

Both \cite{benson2010network, kandula2009nature} analyze traffic patterns in data centers. These measurement results help in developing an understanding of the nature of traffic in an operational data center (especially for those without direct access). The measurements results have also guided many system designs. VL2 \cite{greenberg2009vl2} is one such example; the validation of Valiant Load Balancing on flow level is based on the small flow traffic pattern observed in a data center. Our measurement is not meant to address the latency within data center. Instead, we start from the experience of end users, and partition the latency of queries to cloud-based services into the WAN and DC portions. Such a partition can provide additional insight as to the importance of deadlines in the data center as has been previously studied in \cite{wilson2011better}. By comparing the portion of time spent in the WAN and the DC, we aim to understand which is the primary element that determines an end-host's experienced RTT.

There are also several works on mobile networks \cite{xu2011cellular, huang2011mobiperf} that try to infer the networking conditions by distributing their mobile applications to a large numbers of users. These work provides insight on a larger scale, but seldom touch on the latency dynamics for a single user over a longer time period. Our measurement aims to fill this gap. In addition, most of our cellular network measurements occur within the 4G-LTE network, which can reflect the relatively-recent mobile networking development.

%\subsection{Measurement Methodologies}
%\label{sec:meas-meth}

% Early Vern Paxson's work in \cite{paxson1997measurements, paxson1997end, paxson2004strategies} serves as guidelines in our design and implementation of all measurement experiements.

%%% Local Variables: 
%%% mode: latex
%%% TeX-master: "main"
%%% End: 

%\section{Expected Results}
\label{sec:results}

We expect to have results in the form of actual measurement tools and data, as well as analysis tools and reproducible analysis output. The dataset will consist of latency measurements between end hosts along with meta-data (the DNS query, Geo-IP resolution, time of testing, environment of testing, etc.). Our analysis will compare latency to time and physical location, and separate WAN latency from datacenter latency. Various visualization might be employed in assisting answering those questions in Sec.\,\ref{sec:introduction}.

%%% Local Variables: 
%%% mode: latex
%%% TeX-master: "main"
%%% End: 


\section{Methodology}
\label{sec:methodology}


%%% Local Variables: 
%%% mode: latex
%%% TeX-master: "main"
%%% End: 


\section{Experiment}
\label{sec:experiment}

The data presented was collected between 4/20/13 to 4/28/13, and is composed of approximately 4.95 million successful measurements.

\subsection{Turbo-King Implementation}
\label{sec:turbo-king}

\subsubsection{Measurement Node}
We utilized the dnspython libraries to implement our Turbo-King variant. Each measurement node runs a Turbo-King server (composed of a DNS client and server) which exposes a remote procedure call service using the RPYC library. We deployed the Turbo-King server to 56 Planet Lab nodes (fig x).

\subsubsection{Central Name Server}
Our central name server is implemented using the Python twisted framework, which allowed for highly parallel and asynchronous handling of DNS queries. We deployed 2 nameservers on Amazon EC2.

\subsubsection{Controller}
Latency measurements were centrally managed from a controller running on EC2. The controller is responsible for issuing remote-procedure calls to Turbo-King servers and storing the resulting data. The controller randomly selects 2 target name servers, the first from the set of open recursive resolvers, and the second from the set of all resolvers. It then determines the 10 nearest Turbo-King servers to the first target, issues a remote procedure call to each, and stores the results in a relational database. The controller is composed of multiple threads that follow this procedure and run continuously.

%%% Local Variables: 
%%% mode: latex
%%% TeX-master: "main"
%%% End: 


\section{Analysis}
\label{sec:analysis}


%%% Local Variables: 
%%% mode: latex
%%% TeX-master: "main"
%%% End: 


\section{Discussion}
\label{sec:discussion}


%%% Local Variables: 
%%% mode: latex
%%% TeX-master: "main"
%%% End: 


\section{Conclusion and Future Work}
\label{sec:conclusion}

We believe that our results are relevant in displaying the current state of latency in wide-area, data center, and mobile networks. It is our goal to continue collecting measurements and refine our methodologies. Our next step is to compare our wide-area latency results to existing datasets, such as iPlane. For our DC measurements, we plan on re-evaluating our querying strategy; we hope to develop a better method of generating query strings as well as a long term solution to overcoming the limit on query rate. Finally, we hope to expand our mobile network study by measuring full application layer tasks, such as HTTP requests.

%%% Local Variables: 
%%% mode: latex
%%% TeX-master: "main"
%%% End: 


% \section{Acknowledgments}

\bibliographystyle{abbrv}
\bibliography{Measurement}

\end{document}
