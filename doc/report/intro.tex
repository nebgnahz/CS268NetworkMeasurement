\section{Introduction}
s\label{sec:introduction}

% why do measurement
The measurement of Internet has become increasingly challenging due to its complexity, large-scale, heterogeneity and variability. However, the importance is also increasing because an understanding of the performance of the Internet architecture will provide the insight necessary to design a better architecture, which is either incrementally deployable or a clean slate design. Depending on figure of interest, there exist many metrics to characterize the performance -- connectivity, available bandwidth, latency, etc. Various tools have been proposed to conduct these measurements (see Sec.\,\ref{sec:related-work} for a complete review). This work does not aim to introduce a revolutionary methodology to conduct measurements. Instead, we base our work on many existing tools and available infrastructure. We will discuss both our methodology and an analysis of the measurements we have collected.

% split up to three parts
Our work is composed of three distinct parts: the latency in the WAN, latency as experienced by end-users when interacting with the DC, and latency within cellular networks.

The goal in measuring WAN latency is to develop and understanding of how latency is related to distance. The traditional model used to calculate the latency between end-hosts is to sum propagation delay, transmission delay, queuing delay and processing delay \cite{kurose2001computer}. The propagation delay is determined by the length of cables (for wired networks), and will have the lower bound, which we refer to ``speed-of-light limit'' in this paper. The transmission delay and processing delay depend on the speed of routers (or switches) and the size of the packet. The general trend is that these delays are getting smaller. Nonetheless, queuing delay has a lot of variation depending on the traffic load; in the worst case, packets may even be dropped. In our study we aim to uncover the current state of latency in the WAN given the new and recent developments in networking hardware and software. Furthermore, we seek to make a comparison to the theoretical speed-of-light limit. In our study we have adopted the King/Turbo-King \cite{gummadi2002king, leonard2008turbo} methods and deployed measurement nodes to PlanetLab \cite{chun2003planetlab} servers located globally. In a continuous study spanning more than 10 days we have obtained approximately 5 million measurements for analysis.

In the second section of our study, we analyze cloud services and data center performance from the end-users' perspective. More specifically, we have chosen to perform a study of Google Search due to its popularity and Google's promise of delivering timely responses to end-users. Furthermore, we are motivated to use Google Search because of the fact that the return page from Google includes an estimate of time spent inside the DC. By inspecting the TCP trace of a query conversation, we found out that this estimated time is a fairly accurate portrayal of time spent within the data center. Because each transaction involves several rounds of data transmission, the round-trip time (RTT) will influence the overall query time. We also use \texttt{ping} to measure the network latency.
%We extract Google's IP address from the packet trace using \texttt{tcpdump} and utilize \texttt{ping} to query this IP to measure the network latency. 
By examining such data from various vantage points (deployed to PlanetLab servers), we would partition the latency and provide insights for network optimizations.

The final aspect of our study is an analysis of cellular networks, and in particular the relatively new and growing 4G-LTE deployments. By adopting a flat IP architecture, carriers promise to reduce the network latency and decouple radio access from core network evolution. To measure the real performance of the network, we use \texttt{traceroute} to examine the latency on each hop and the routing stability within the cellular backbone.

% summarize the contribution
To summarize, we make the following contributions in this measurement study:
\begin{itemize}
\setlength{\itemsep}{1pt}
\setlength{\parskip}{0pt}
\setlength{\parsep}{0pt}
\item We present measurements of the wide-area network, the fraction of round-trip time spent within the data center, and latencies within the cellular network backbone.
\item We provide the methodology and analysis of our collected measurements.
\end{itemize}

% rest of the paper
In the rest of this paper, we will first describe the related works in Sec.\,\ref{sec:related-work}. And then we detail our measurements and analysis for WAN, DC and cellular networks in Sec.\,\ref{sec:latency-wan}, Sec.\,\ref{sec:latency-DC} and Sec.\,\ref{sec:latency-in-cellular} respectively. We discuss our experience in conducting this measurements in Sec.\,\ref{sec:discussion}. Sec.\,\ref{sec:conclusion} concludes the paper and discusses some of the future work.

% In the rest of this paper, we will first describe the related works in Sec.\,\ref{sec:related-work}. And then detail our measure methodology in Sec.\,\ref{sec:methodology}. The experiment setup is in Sec.\,\ref{sec:experiment}, which is followed by the analysis of the data in Sec.\,\ref{sec:analysis}. We then briefly discuss a few questions regarding our measurements in Sec.\,\ref{sec:discussion}. Sec.\,\ref{sec:conclusion} concludes the paper and discusses some of the future work.

%%% Local Variables: 
%%% mode: latex
%%% TeX-master: "main"
%%% End: 
